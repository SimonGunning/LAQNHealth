% Chapter 5

\chapter{Results} % Main chapter title

\label{Chapter5} % For referencing the chapter elsewhere, use \ref{Chapter5} 

%----------------------------------------------------------------------------------------

% Define some commands to keep the formatting separated from the content 


%----------------------------------------------------------------------------------------
\section{Data Description}
The dataset that is produced from the analysis phase is called HealthPollutionDS and it consists of 378 rows. Each row has: Ward, WardCode, Register, ListSize, Prevalence, Year, Monitor, Longitude, Latitude, NO2, PM10 and DUST. The longitude and latitude describe the position of any monitor within the ward. There are missing readings for some pollutants for some monitors. There can be a variation in the size of the population of a Ward so the prevalence of Asthma amongst patients on the List is calculated.




\section{Trend Graphs}
The analysis covers 5 years, from 2012 to 2016. It only covers the GP Practices in London. During that time there has been a small decline in the prevalence of Asthma as shown in Figure \ref{fig:AsthmaPrevalence}. This graph shows annual average levels for all of London as a percentage of patients that are on the Practice list that are recorded as suffering from Asthma. The range is from 2.0 percent to 9.2 percent.

\begin{figure}
\centering
\includegraphics[width=\textwidth]{Figures/AsthmaPrevalenceTrendGraph}
\decoRule
\caption[Asthma Prevalence]{Asthma prevalence over time.}
\label{fig:AsthmaPrevalence}
\end{figure}



There has also been a small decline in the levels of NO2 and PM10 pollutants as recorded by the monitors. This is shown in Figure \ref{fig:PollutionTrend}.
The graph shows the annual mean levels of the NO2 and PM10. These pollutants are the most commonly measured by the LAQN network. There are 330 readings in the dataset for NO2 in a range from 20 to 185 ug/m3 . There are 286 readings for PM10 in a range from 12 to 43 ug/m3 .

\begin{figure}
\centering
\includegraphics[width=\textwidth]{Figures/PollutionTrendGraph}
\decoRule
\caption[Pollution trend]{Pollution trend over time}
\label{fig:PollutionTrend}
\end{figure}

\section{Maps}
The following choropleth maps were generated from the monitor files and the aggregated QOF files. 
See  \ref{fig:2012NO2} , \ref{fig:2013NO2} , \ref{fig:2014NO2} ,\ref{fig:2015NO2} ,\ref{fig:2016NO2} .
The maps also list the top 3 wards for pollution and the top 3 wards for Asthma prevalence. These maps are best viewed on a website or using the output files from matplotlib rather than in a printed document. This allows the user to zoom in.

A visual inspection of the graphs shows that the monitors (in red) are not uniformly distributed throughout London. It also shows that the wards which have the highest incidence of Asthma do not correspond with the monitors that have the highest readings.There also appears to be a cluster of wards (around inner west London) that have low levels of asthma but a large number of monitors with relatively high pollution readings. This is where the maps may suggest that we could look for alternative predictors. For example, this is a relatively wealthy area of London and so it may be worth investigating the relationship between income and asthma prevalence.


%\makeatletter
%\setlength{\@fptop}{0pt}
%\setlength{\@fpbot}{0pt plus 1fil}
%\makeatother

\begin{figure}[]
	\centering
	\hspace*{-2cm}\includegraphics[scale=0.55, angle=90]{Figures/2012NO2}
	
	%\decoRule
	\caption[2012NO2]{2012 NO2}
	\label{fig:2012NO2}
\end{figure}
\begin{figure}[]
	\centering
	\hspace*{-2cm}\includegraphics[scale=0.55, angle=90]{Figures/2013NO2}
	
	%\decoRule
	\caption[2013NO2]{2013 NO2}
	\label{fig:2013NO2}
\end{figure}

\begin{figure}[]
	\centering
	\hspace*{-2cm}\includegraphics[scale=0.55, angle=90]{Figures/2014NO2}
	
	%\decoRule
	\caption[2014NO2]{2014 NO2}
	\label{fig:2014NO2}
\end{figure}
\begin{figure}[]
	\centering
	\hspace*{-2cm}\includegraphics[scale=0.55, angle=90]{Figures/2015NO2}
	
	%\decoRule
	\caption[2015NO2]{2015 NO2}
	\label{fig:2015NO2}
\end{figure}
\begin{figure}[]
	\centering
	\hspace*{-2cm}\includegraphics[scale=0.55, angle=90]{Figures/2016NO2}
	
	%\decoRule
	\caption[2016NO2]{2016 NO2}
	\label{fig:2016NO2}
\end{figure}
%------------------------------------------------------------------------------------------
\section{Regression results}
The scatter plot graph at \ref{fig:Scatterplot} shows the NO2 pollution levels mapped against the asthma prevalence.


\begin{figure}
	\centering
	\includegraphics[width=\textwidth]{Figures/NO2Scatterplot}
	\decoRule
	\caption[Scatterplot]{Scatterplot of NO2 and Asthma prevalence}
	\label{fig:Scatterplot}
\end{figure}
I created new datasets for a single pollutant (NO2) and the asthma prevalence. All the rows with no pollution readings were dropped.

\subsection{Linear Regression}

The following results were achieved by running a linear regression using the asthma prevalence as the dependant variable and NO2 concentration as the independent variable.
The coefficient of regression is -0.00012064 indicating a very small decline in asthma prevalence for each unit increase in pollution. The Rsqared value is -0.16412615624341975 . This indicates the linear model does not fit the data. The results for the linear regression are not good. The fact that the dependant variable is a percentage means that the linear regression model is not the best choice as it expects the dependant variable to be unbounded.

\subsection{Logistic Regression}
Therefore I ran a Logistic regression against the same data.
I used the statsmodels Python library to create the logistic regression model. This produced the following model.

\begin{verbatim}
Logit Regression Results                           
==============================================================================
Dep. Variable:                      y   No. Observations:                  310
Model:                          Logit   Df Residuals:                      309
Method:                           MLE   Df Model:                            0
Date:                Sun, 16 Sep 2018   Pseudo R-squ.:                     inf
Time:                        12:04:47   Log-Likelihood:                -24.274
converged:                       True   LL-Null:                        0.0000
LLR p-value:                       nan
==============================================================================
               coef    std err          z      P>|z|      [0.025      0.975]
------------------------------------------------------------------------------
x1            -0.0631      0.006    -10.378      0.000      -0.075      -0.051
==============================================================================

\end{verbatim}

The negative coefficient indicates an inverse relationship between pollution and disease prevalence.
The P value is low (0)indicating that pollution is a predictor for disease prevalence, albeit in a negative way.
This is not the result that intuitively we would expect. I will discuss possible confounding factors in the conclusion.


\subsection{Spatial Regression}
Adding in a correction for spatial autocorrelation will tend to reduce the effect a change in the independent variable has on the dependant variable in the model. So the gradient is expected to be lower. I have used the pysal library for the regression. Unfortunately there is no Logistic Regression function available in the current version, so I have used the OLS function. This has the drawback previously discussed. In order to get the process to work I have had to run it on a subset of the rows of the complete dataset. Thus the results are only indicative. The gradient is -0.0002223 which is slightly more negative than the non-spatial regression. This can be explained by the reduction in rows.

Of interest is the Morans I measurement, which is -0.423497 . This number ranges from -1 to +1 . A negative number indicates that events in the dataset are not spatially autocorrelated. This can be explained by the distribution of the monitors in the wards in London. There are fewer monitors than wards.


\begin{verbatim}
SUMMARY OF OUTPUT: ORDINARY LEAST SQUARES
-----------------------------------------
Data set            :     unknown
Weights matrix      :     unknown
Dependent Variable  :     dep_var                Number of Observations:          18
Mean dependent var  :      0.0429                Number of Variables   :           2
S.D. dependent var  :      0.0098                Degrees of Freedom    :          16
R-squared           :      0.2010
Adjusted R-squared  :      0.1511

------------------------------------------------------------------------------------
Variable     Coefficient       Std.Error     t-Statistic     Probability
------------------------------------------------------------------------------------
CONSTANT       0.0543825       0.0061210       8.8845271       0.0000001
var_1      -0.0002223       0.0001108      -2.0061837       0.0620469
------------------------------------------------------------------------------------

DIAGNOSTICS FOR SPATIAL DEPENDENCE
TEST                           MI/DF       VALUE           PROB
Lagrange Multiplier (lag)         1           0.000           0.9918
Robust LM (lag)                   1           0.117           0.7318
Lagrange Multiplier (error)       1           1.729           0.1885
Robust LM (error)                 1           1.846           0.1742
Lagrange Multiplier (SARMA)       2           1.846           0.3972

================================ END OF REPORT =====================================
Morans I
Morans I   -0.423497
Z-Score    -1.278838
P-Value     0.100477
	content...
\end{verbatim}


