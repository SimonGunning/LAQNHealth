% Chapter 6

\chapter{Conclusion} % Main chapter title

\label{Chapter6} % For referencing the chapter elsewhere, use \ref{Chapter1} 

%----------------------------------------------------------------------------------------

% Define some commands to keep the formatting separated from the content 


%----------------------------------------------------------------------------------------

\section{Analysis}

An inspection of the maps reveals that the positioning of the monitors is not uniform. Also there seems to be a wide range of readings from the monitors. It is likely that some are positioned on very busy roads and others in more quiet locations. Furthermore it is observable that in some locations there is a high level of pollution and a relatively low incidence of asthma. There also appear to be wards that have very high levels of asthma, unfortunately they do not have monitors. Some of the wards with high levels of asthma are not in central London where there may be more pollution. It appears that there may other factors that could predict asthma. The maps are useful for suggesting geographical areas to investigate. Other predictors could be smoking, wealth or the number of children in the ward. More children than adults may be registered in the practices.

The trend graphs indicate a small reduction in the levels of pollution and a small reduction in the prevalence of asthma over the five year period. This is across the whole of London. However the results of the regressions do not mirror this at the war/monitor level. In fact they all indicate a very small inverse relationship between pollution and asthma. Our understanding of the domain leads us to think that this is a misleading result. This could be because the geographic ward is not the best level to run the analysis with. It may be the case that the dispersal of pollutants happens in such a short distance that trying to measure the effects at the level of a ward are inappropriate. It is also the case that there are not enough monitors distributed across the wards.

The linear regression does not appear to be the best approach. The logistic regression approach is a better model. Ideally this would be combined with the spatial modelling. 

There is not sufficient evidence to dismiss the null hypothesis (H0). Which is that there is no association between high levels of pollution as measured by the monitoring system and a high prevalence of Asthma at a ward level in the surrounding area of a monitor. However given that there is a generally acknowledged association between traffic pollution and asthma it is worth considering other approaches. 

\section{Next Steps}
An adjustment could be made to the study. For each monitor the closest GP Practice could be located. Then these figures could be analysed. This could be done using the latitude and longitudes of the practices and the monitors. This would be a better approach if the dispersal of pollutants occurs in a short distance. However, there are still many other factors that would need to be accounted for such as wind speed and direction.